
\subsection{One Example}
\label{subsec:shillerpound}

Here, we provide a first example of an economic question that has been formulated in a thoroughgoing epidemiological way.  Our purpose, at this point, is neither to defend this way of doing things nor to extract economic insights -- we do both in section~{\ref{subsec:assetprice}} below -- but simply to illustrate how the epidemiological toolkit can be deployed.

\href{https://github.com/iworld1991/EpiExp/blob/master/Literature/shiller1989survey.pdf}{\cite{shiller1989survey}}\footnote{This paper builds on the earlier work comparing the efficient market hypothesis of stock prices and an alternative model incorporating social dynamics \href{https://github.com/iworld1991/EpiExp/blob/master/Literature/shiller1984stock.pdf}{\citep{shiller1984stock}}. } use an SIR model to capture how the interest in particular stocks spreads in a population; we examine a model almost identical to theirs.\footnote{Our treatment makes two inconsequential modifications.  First, in order to be able to instantiate the model using the \href{https://ndlib.readthedocs.io/en/latest/}{\texttt{NDLib}} computational toolkit described below, we rewrite the originally continuous-time model in a discrete-time form. Second, the original paper described an additional stochastic shock to the change in $I_t$ meant to capture a potential ``change in the `source' of the infection or the nature of the contagion.''  Because that shock was not actually used for any results in the paper, we neglect it in our exposition.}


At any date  $t$, a large population of investors of size $N$ is divided into three ``compartments.''  (See Figure \ref{fig:sir_diagram}).  $I_t$ represents investors who are currently ``infected'' with interest in a certain stock,  $S_t$ corresponds to investors who are not infected but are ``susceptible'' to becoming interested in the stock, and $R_t$ are investors who have been ``infected'' but have ``recovered'' from the infection.\footnote{The ``recovery'' compartment contains investors who have lost interest in the stock.  For our purposes here, we do not need to define the exact consequences of `recovery' -- like, whether it means that the person sells the stock.  See the original paper for further exposition.}


[Insert Figure \ref{fig:sir_diagram} here]


\newcommand{\contactNum}{\chi}
\newcommand{\tranProb}{\tau}
\newcommand{\exposures}{\mathcal{E}}

In each period, each person is expected to have contact with $\contactNum>1$ others, randomly selected from the entire population (this is the `random mixing' assumption mentioned above).  In the SIR framework, the only kind of contact with any consequence is between an infected person and a susceptible person: Such an encounter has a probability $\tranProb$ of causing the susceptible person to become infected.

Epidemiological models typically define a parameter $\beta$ that combines consequences of the rate of social connection $\contactNum$ and the rate of transmission $\tranProb$:
\footnote{In any extended SIR model embedding an explicitly defined connection network via which the ``disease" spreads, the value of $\beta$ is equal to the product of the average number of connected nodes (``degree'' in the terminology of network theory), and the infection probability conditional on the contact. For instance, in a random graph (\cite{erdos1960evolution})  with connection probability $p$ and the size of network N, the average contacts every agent has is $(N-1)p$. See \cite{newman2002spread} and  \cite{jackson_social_2010} for the results from an SIR model augmented with various social networks.}

\begin{verbatimwrite}{./Equations/beta}
\begin{equation}
	\label{eq:beta}
    \beta  = \tranProb \contactNum
\end{equation}
\end{verbatimwrite}
\begin{equation}
	\label{eq:beta}
    \beta  = \tranProb \contactNum
\end{equation}


The expected number of new infections generated in period $t$ (corresponding to the decline in the number of susceptible persons) can now be calculated transparently: A fraction $S_{t}/N$ of an infected person's contacts will be susceptible, so the number of newly generated infections per infected person will be $\tranProb \times \contactNum \times (S_{t}/N).$

The population of infected persons also changes: Every infected person recovers with a probability of $\gamma$ per period.  Putting these elements together, the  changes in the population in different compartments are given by
\begin{equation}
	\label{eq:sirdyn}
	\begin{split}
	&	\Delta S_{t+1} = -\beta I_{t}(S_{t}/N) \\
	&	\Delta I_{t+1} = \beta \frac{S_{t}}{N}I_{t} - \gamma I_t \\
&		\Delta R_{t+1} = \gamma I_t
	\end{split}
\end{equation}

%The term $\beta \frac{S_{t}}{N}I_{t}$ captures the number of people who ``flow'' from ``compartment $S$'' to ``compartment $I$'', which is proportional to the infection rate $\beta$, the fraction of people who are susceptible $\frac{S_t}{N}$, and the number of the infected $I_t$. The  term $\gamma I_t$ captures the number of people who ``flow'' from $I$ to $R$

The simplest special case of the SIR model is one with a recovery rate of $\gamma=0$, in which case the model reduces to the simple SI model discussed in Section \ref{subsec:epi_framework}.  Another straightforward case is $\beta < \gamma$, in which from any starting point the population of infected persons $I$ gradually dies down to zero.

\newcommand{\Rzero}{\mathcal{R}(0)}

The interesting cases emerge when the `basic reproduction ratio' $\Rzero = (\beta/\gamma)$ exceeds one (this $\Rzero$ is unrelated to the $R$ used elsewhere to measure the recovered population), because $\Rzero > 1$ guarantees that an initial arbitrarily small infection will grow, at least for a while (assuming that at the beginning everyone is susceptible, $S_{0}/N = 1$).

To illustrate the model's implications in such a setting, we parameterize the model with four such combinations of parameter values taken from \cite{shiller1989survey}, characterizing two different kinds of investors and two categories of stocks.  (Section~\ref{subsec:assetprice} describes the investors and stock categories, and interprets the economics; here we confine our observations to the epidemiology.)

We explore the quantitative implications using one of the many computational toolkits for analyzing such models that have proliferated in recent years.\footnote{Specifically, we use the Python library NDlib (\cite{rossetti2018ndlib}) for the simulation of the SIR model here. The library builds upon another Python library called NetworkX (\cite{hagberg2008exploring}), a toolkit for analyzing complex networks.}  The toolkit we use lets users specify explicitly the network structure on which the disease spreads. We exploit the fact that a random-mixing SIR model can be approximated with a SIR model residing on an ex-ante generated random graph (\cite{erdos1960evolution}) when the transmission probability $\tau$ and the average number of connections $\chi$ in the graph are configured such that their product is equal to the calibrated value of infection rate $\beta$. (Equation \ref{eq:beta})\footnote{See the companion \href{https://github.com/iworld1991/EpiExp/blob/master/Python/SIR_Ndlib.ipynb}{Jupyter Notebook} of this paper for detailed implementation.}

As plotted in Figure \ref{fig:sir_simulate}, the proportions of Susceptible (solid line), Infected (dash line), and Recovered (dash-dot line) investors are depicted on the vertical axis, and elapsed time since the initial date of infection is on the horizontal axis.  Also plotted is the limiting size of the recovered compartment, for which an analytical solution exists.\footnote{Given a constant basic reproduction ratio $\beta/\gamma$ that is strictly greater than $1$, and an initial fraction $S_{0}/N$ close to 1, there exists a limiting size of each compartment as time goes to infinity. The limiting fraction of $R$, denoted as $r_{+\infty} = R_{+\infty}/N$, is the solution to the implicit equation: $e^{-\frac{\beta}{\gamma} r_{+\infty}} = 1-r_{+\infty}$.  In the limit, the infected compartment is of size $I_{\infty}=0$.  See  \href{https://en.wikipedia.org/wiki/Compartmental_models_in_epidemiology\#Transition_rates}{this Wikipedia page}, \cite{harko2014exact}, \href{{https://iopscience.iop.org/article/10.1088/1751-8121/abc65d}}{\cite{kroger2020analytical}}, \cite{okabe2021microscopic} for details of the results.}

Two common patterns emerge from the simulation under these four sets of parameters of infection and recovery rates.  First, since in all four cases the basic reproduction ratio $\Rzero$ is greater than 1, in all four cases there is an outbreak. The size of the infected population first expands to its maximum value and then gradually levels off to zero, exhibiting a hump-shaped ``viral curve'' commonly seen   SIR model.  Second, in all scenarios, the system ultimately converges to a steady-state where most of the people have cycled through infection and recovery, with a small proportion remaining susceptible. Even in the smallest reproduction ratio, the proportion who cycle through the process of Infection and Recovery is almost 85 percent, implying a high degree of infectiousness. Under other configurations, the limiting size of $R$ is close to 100 percent.

The main difference in the parameterizations is the speed with which these eventualities play themselves out, which varies considerably.  (Since we are not interpreting the model in economic terms here, the differences we are interested in are only the relative proportions and not the absolute time intervals).

%Even using the results of their own surveys explicitly designed for the purpose, \cite{shiller1989survey} needed to exercise considerable ingenuity to produce the calibrations we have used above.

We highlight the paper here because it presents an example that satisfies all our criteria for an epidemiological model of economic expectations. First, it articulates and a explicit structural mathematical mechanism by which an idea (in this case, interest in a stock) spreads in the population as a result of social communication. Second, the model has clear assumptions and predictions for both the micro and macro dynamics of expectations, which can in principle be tested (or calibrated) with measurable data.  Third (as we explain below), dynamics of separately measurable economic phenomena (stock prices) are hypothesized to be a consequence of the dynamics of those expectations.  Not many papers in the large literature satisfy all these criteria.

