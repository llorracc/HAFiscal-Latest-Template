
\subsection{Diffusion of Technology}\label{subsubsec:techDiffusion}

\begin{center}
	[Insert Figure \ref{fig:graph_diffusion}  here]
\end{center}

%Economists have understood since \cite{solow1956contribution} that technological progress is the wellspring of economic growth.  Although in studies of the process by which fundamentally new technical knowledge is generated, we are not aware work that could be classified as epidemiological, the vast bulk of technological progress for most of the individual agents who are progressing does not reflect their independent invention of ideas novel to humanity -- it reflects their adoption of knowledge invented by others.  In recognition of this fact, an extensive literature has studied a topic usually identified as `the diffusion of technology.'  Traditionally, this line of work has been seen as separate from the domain of economic expectations.  But the reason for adopting a new technology is surely that there is an expected gain from doing so, a point that is explicitly made at various places in the more theoretically-minded branches of the literature.

\href{https://github.com/iworld1991/EpiExp/blob/master/Literature/arrow_classificatory_1969.pdf}{\cite{arrow_classificatory_1969}} was one of the earliest papers in economics to draw an explicit analogy between the diffusion of ideas and the spread of disease.  He puts interpersonal communication at the center of knowledge diffusion and the consequent economic growth, and argues that the speed of knowledge diffusion may account for levels and dynamics of international differences in income.  (See Section~\ref{subsec:nonecon} for earlier work, on which Arrow draws, about the diffusion of scientific knowledge.)

He conjectures that the speed of knowledge diffusion is influenced by factors that he explicitly compares to  those that influence the spread of disease including (1) the perceived reliability of the sender (which affects infectiousness); (2) socioeconomic traits (which affect exposure and susceptibility); (3) the understandability of information by the receiver (degree of immunity); and so on.

Arrow's interpretation is the step that puts this topic squarely in the realm of EE modeling, under the mild further assumption articulated above: That what spreads is the `expectation' that adoption of the technology will yield higher productivity (an expectation that was not originally measured because the seminal research predated the era when expectations were solicited on surveys; but expectational questions of exactly this kind have been asked in more recent work on diffusion, see~\cite{banerjee2013diffusion}, and unsurprisingly confirm that people adopt a technology when they expect it will be beneficial).

\href{https://pubmed.ncbi.nlm.nih.gov/23888042/}{\cite{banerjee2013diffusion}} observe the real-world network and pattern of the diffusion of microfinance in a number of Indian villages. The paper provides direct evidence for word-of-mouth diffusion through a social network. What is novel about the model compared to a canonical epidemiological model is that it differentiates the agents who simply adopt the technology because they have heard about it from others (an `information passing mechanism') and those who have adopted due to others' participation (an `endorsement mechanism'). This can be seen as an example of how standard epidemiological models can be extended to incorporate alternative infection rules to accommodate more sophisticated applications.


For a broader survey of how alternative epidemiological models of technological diffusion generate different shapes of  ``adoption curves'' with consequent effects on the path of economic growth, see  \href{https://github.com/iworld1991/EpiExp/blob/master/Literature/young2009innovation.pdf}{\cite{young2009innovation}}, who shows that how the shape of diffusion curves differs in models of `inertia,' `social influence,' `social learning,' and a standard SIR model.\footnote{It is worth pointing out that here we do not survey a large parallel literature on technology/innovation diffusion in economics that features the role of social learning, as this work is not explicitly built upon epidemiological frameworks. Examples of such include  \href{https://www.researchgate.net/publication/222676428_Social_Learning_in_a_Heterogeneous_Population_Technology_Diffusion_in_the_Indian_Green_Revolution}{\cite{munshi2004social}},   \href{https://www.jstor.org/stable/41038754}{\cite{comin2010exploration}} and so on.}  % the first one is field evidence on tech diffusion in Indian showing that information flow is weaker when  underlying hterogeneity is higher.  The scond  builds the process of tech diffusion  into a neocalssical growth model.

Not only are mechanisms of the spread of technology and disease comparable, they may interact.  \href{https://github.com/iworld1991/EpiExp/blob/master/Literature/fogli2012germs.pdf}{\cite{fogli2021germs}} develop a model in which the structure of the networks connecting people (`nodes') allows the authors to explore the roles of three dimensions that have emerged in as central to the network theory literature that has developed since the pioneering work of \cite{erdos1960evolution}.  One is `degree' -- the number of other people with whom a person is directly connected.  A second measure aims to capture the extent of `clustering': Roughly, the extent to which my friends know each other.  The third is the extent to which people have connections to others who are randomly selected in the broader population.  In their model, both productivity and disease spread through these connections, and as a result the dynamics of productivity and disease are connected.  For example, although the authors do not put it in quite this way, one implication of the model is that a bright side of the spread of disease is that the deceased are replaced by higher-productivity nodes.%  (The authors perhaps do not intend for the reader to take all the paper's policy implications at face value.)
