\subsection{Non-economic applications of epi models}\label{subsec:nonecon}\hypertarget{nonecon}{}

This section highlights some interesting applications of epidemiological models to fields other than economics.

We focus on the following three areas:
\begin{verbatimwrite}{./Slides/NonEcon}
\begin{enumerate}
    \item the spread of news, fake news, and rumors
    \item the diffusion of scientific ideas
    \item the dissemination pattern of internet content such as memes
\end{enumerate}
\end{verbatimwrite}
\begin{enumerate}
    \item the spread of news, fake news, and rumors
    \item the diffusion of scientific ideas
    \item the dissemination pattern of internet content such as memes
\end{enumerate}


The first epidemiological model we have been able to find in which rumors spread like disease is by \cite{daley1964epidemics}, whose work spurred a subsequent literature that explored variants of the model allowing for different `compartments.'  A highly cited example is a paper by \cite{jin2013epidemiological} that augments a model with the usual three compartments of Susceptible, Infected, and Exposed with another compartment of skeptics, and estimates the model with a diffusion pattern of eight real events among Twitter users, including news events such as the Boston Marathon Bombings, resignation of the Pope, and rumors such as Doomsday, Obama injured, etc. (See Figure \ref{fig:news_curve}). The augmented model with estimated parameters matches the dynamics of the news/rumor reasonably well.

\begin{center}
	[Insert Figure \ref{fig:news_curve}  here]
		\end{center}

	Other empirical studies on news spreading are not with a structural model of epidemiology but provide useful information for modeling choices. For instance, as to the features of the information affecting its infectiousness,  \href{https://science.sciencemag.org/content/359/6380/1146}{\cite{vosoughi_spread_2018}} found that falsehood spreads faster than the truth possibly because of its novelty and emotional arousal. This suggests that epidemiological modeling shall differentiate the nature of information/news.  %About 126,000 rumors were spread by ∼3 million people. False news reached more people than the truth; the top 1% of false news cascades diffused to between 1000 and 100,000 people, whereas the truth rarely diffused to more than 1000 people. ...We found that false news was more novel than true news, which suggests that people were more likely to share novel information. Whereas false stories inspired fear, disgust, and surprise in replies, true stories inspired anticipation, sadness, joy, and trust. Contrary to conventional wisdom, robots accelerated the spread of true and false news at the same rate, implying that false news spreads more than the truth because humans, not robots, are more likely to spread it.

	There is economic research that tackles the question of fake news and misinformation from other angles but contains insights for the modeling choices. For instance, 	\href{https://github.com/iworld1991/EpiExp/blob/master/Literature/allcott2017social.pdf}{\cite{allcott2017social}} used a post-2016 election survey of 1200 U.S. adults to analyze the importance of social media on fake news consumption, exposure to fake news as well as partisan composition. The paper highlights social media as one of the important (but not dominant) channels for the diffusion of fake news.  The paper also constructs a model with the supply side of the fake news encompassing profit-maximizing entities appealing to consumers subject to confirmation bias. This seems a very natural extension of standard epidemiological models to incorporate the production side of the information.  By a similar token, \href{https://www.kdd.org/exploration_files/8._CR.10.Misinformation_in_social_media_-_Final.pdf}{\cite{acemoglu2010spread}} builds a model of social learning with ``forceful'' agents who negate information updating from other agents in the network. This may give rise to misinformation in the equilibrium. The insight from this paper for epidemiological modeling is that different agents might be with different contagion rules in the information spreading.

Closely related to Arrow's work on technological progress is work by \href{https://en.wikipedia.org/wiki/Diffusion_of_innovations}{\cite{rogers1962diffusion}}, who popularized a theory of the ``diffusion of innovations'' based on a meta-analysis of early studies of the spread of ideas within many academic disciplines.  The factors that this literature identifies as determinants of the dynamics of diffusion are directly interpretable as corresponding to the ``infectiousness'' of the idea, the degree to which populations are ``exposed'' to the idea, and many of the other elements of the epidemiological frameworks sketched below.


	Very relevant to the technological diffusion literature, epidemiological models are also used to study the patterns of scientific ideas. \href{https://github.com/iworld1991/EpiExp/blob/master/Literature/bettencourt2006power.pdf}{\cite{bettencourt2006power}} estimates an epidemiological model on the spread of Feynman diagrams through the theoretical physics communities. It also augmented the SIR to SEIR where E represents incubator (exposed but not adopted) state and SEIZ model where Z represents skeptics (mutually  exclusive with being infected) due to competing ideas. The paper suggests that introducing skeptics generates an additional steady state of the model where competing ideas coexist. This differs from two-compartment and three-compartment models in which typically the system converges to a single state.

		\begin{center}
			[Insert Figure \ref{fig:science_ideas_curve} here]
		\end{center}

	The virality pattern of internet content such as memes makes it a natural candidate field for epidemiological modeling. For instance, \href{https://github.com/iworld1991/EpiExp/blob/master/Literature/bauckhage2011insights.pdf}{\cite{bauckhage2011insights}} shows that both epidemiological models and log-normal distribution could characterize the growth and decay of famous internet memes(See Figure \ref{fig:memes_curve}).  By a similar token, \href{https://github.com/iworld1991/EpiExp/blob/master/Literature/kucharski2016modelling.pdf}{\cite{kucharski2016modelling}} fits an epidemiological model to the outbreaks of a number of notable internet contagion such as ``ice bucket challenge'' and ``no-makeup selfies'', suggesting a reproduction ratio in the range of 1.9 to 2.5.
	Besides, 	\href{http://cs.stanford.edu/~ashton/pubs/twiral.pdf}{\cite{goel2016structural}} explores the structural factors (independent from the intrinsic properties of the content, nature of contact process, etc.) in which lead to viral spreading patterns.  Based on a billion diffusion events on Twitter, the analysis suggests great diversity across different events, i.e.  infections both from a common source like an influencer and word-of-month via social communications (a la SIR) may trigger viral events.

	The diffusion patterns of internet content with different emotions may offer some guidance for economists modeling economic sentiment contagion, For instance, \href{https://journals.sagepub.com/doi/10.1509/jmr.10.0353}{\cite{berger2012makes}} found that emotional arousal matters for virality of online-content: ``content that evokes high-arousal positive (awe) or negative (anger or anxiety) emotions is more viral. Content that evokes low-arousal, or deactivating, emotions (e.g., sadness) is less viral. '' Moreover,  \href{https://arxiv.org/abs/1805.12512}{\cite{zannettou2018origins}} found that content of memes affect the virality: racist and political memes are the most common type of viral content.

	% Internet memes are increasingly used to sway and manipulate public opinion. This prompts the need to study their propagation, evolution, and influence across the Web. In this paper, we detect and measure the propagation of memes across multiple Web communities, using a processing pipeline based on perceptual hashing and clustering techniques, and a dataset of 160M images from 2.6B posts gathered from Twitter, Reddit, 4chan's Politically Incorrect board (/pol/), and Gab, over the course of 13 months. We group the images posted on fringe Web communities (/pol/, Gab, and The_Donald subreddit) into clusters, annotate them using meme metadata obtained from Know Your Meme, and also map images from mainstream communities (Twitter and Reddit) to the clusters. Our analysis provides an assessment of the popularity and diversity of memes in the context of each community, showing, e.g., that racist memes are extremely common in fringe Web communities. We also find a substantial number of politics-related memes on both mainstream and fringe Web communities, supporting media reports that memes might be used to enhance or harm politicians. Finally, we use Hawkes processes to model the interplay between Web communities and quantify their reciprocal influence, finding that /pol/ substantially influences the meme ecosystem with the number of memes it produces, while \td has a higher success rate in pushing them to other communities.

	What do these non-economic applications of epidemiological models offer for economists?  We draw the connections across these different fields primarily because of the simple recognition that news, rumors, scientific ideas, and internet content all affect our thoughts and economic decisions.

	As to the news, the spreading pattern of news and rumors naturally fits the inquires about the implications of financial and economic news on financial markets and the macroeconomy, as surveyed in the previous sections.   Another example is the studies focus on consumer sentiment.\footnote{\href{https://faculty.smu.edu/millimet/classes/eco6375/papers/carroll\%20et\%20al\%201994.pdf}{\cite{carroll1994does}},
		\cite{doms2004consumer},  \href{https://onlinelibrary.wiley.com/doi/abs/10.1111/ecoj.12605}{\cite{benhabib2019sentiments}}, \cite{gorodnichenko2018social}.} Consumer sentiment has been long recognized as an important driver of economic fluctuations. Therefore, questions such as what type of sentiment is more infectious and how do they spread among economic agents are crucial to understanding the inner workings of the macroeconomic dynamics.

	There is also a methodological insight contained in these studies to economists. With the increasing availability of formal and informal human conversations/writings disseminated in the public sphere research, economists should collect these data and better analyze them, as a fruitful supplement to the traditional form of quantitative data from surveys and statistical agencies. What has gradually emerged as a field under the name of ``narrative economics'' \citep{shiller1995conversation,shiller2017narrative} is a great example of this. There are also a rapidly expanding list of economic research that answers economic inquiries by analyzing a large volume of textual/conversational information from the well developed technique of natural language processing (NLP).\footnote{See \cite{tetlock_giving_2007}, \cite{soo_quantifying_2015}, \cite{gentzkow2019text}, \cite{bybee2020structure}, \cite{ash2021text}.}

		\begin{center}
			[Insert Figure \ref{fig:memes_curve} here]
		\end{center}



		% quote: "suggested diversity in patterns in viral content between common-source models or decentralized interpersonal infection models, based on an analysis billion diffusion events on Twitter, including the propagation of news stories, videos, images, and petitions. ". abstract: Viral products and ideas are intuitively understood to grow through a person-to-person diffusion process analogous to the spread of an infectious disease; however, until recently it has been prohibitively difficult to directly observe purportedly viral events, and thus to rigorously quantify or characterize their structural properties. Here we propose a formal measure of what we label “structural virality” that interpolates between two conceptual extremes: content that gains its popularity through a single, large broadcast, and that which grows through multiple generations with any one individual directly responsible for only a fraction of the total adoption. We use this notion of structural virality to analyze a unique dataset of a billion diffusion events on Twitter, including the propagation of news stories, videos, images, and petitions. We find that across all domains and all sizes of events, online diffusion is characterized by surprising structural diversity. Popular events, that is, regularly grow via both broadcast and viral mechanisms, as well as essentially all conceivable combinations of the two. Correspondingly, we find that the correlation between the size of an event and its structural virality is surprisingly low, meaning that knowing how popular a piece of content is tells one little about how it spread. Finally, we attempt to replicate these findings with a model of contagion characterized by a low infection rate spreading on a scale-free network. We find that while several of our empirical findings are consistent with such a model, it does not replicate the observed diversity of structural virality.


