\subsection{Financial Markets}\label{subsec:assetprice}

\begin{center}
	[Insert Figure \ref{fig:graph_investment}  here]
\end{center}

Standard models in finance assume investors choose stocks based on well-informed beliefs about future returns. Social communication plays no role (or none that is explicitly modeled).

To test whether this corresponded to the way actual investors would describe their process of choosing investments, \cite{shiller1989survey} constructed survey questions designed to understand the sources of information that motivated investors' initial interest in the stock that they had most recently purchased (which they designate as `randomly selected' -- \texttt{RAND}).  About a third indicated that their interest in that stock originated with ``a person who is not an investment professional.''  The authors identify another category of stocks owned by their survey respondents as ``rapidly rising'' and for those they find that roughly half of the initial interest in the stock originated with nonprofessionals.  Using a different methodology to designate `randomly selected' versus `rapidly rising' -- `\texttt{RPI}' -- stocks for institutional investors' they find that 10 percent and 30 percent of their initial interest originated from `nonprofessionals.'

Using the data from their survey, estimates of the parameters of their epidemiological model for both individual (`\texttt{IND}') and institutional (` \texttt{INS}') investors reveal considerable heterogeneity in infection rates both within and between the two groups. They also suggest that the infectiousness differs between a randomly selected stock \texttt{RAND} in the sample and a rapidly rising stock \texttt{RPI}. Interestingly, they find that the \texttt{RAND} category is more ``infectious'' than the rapidly rising stock; they propose, plausibly, that public news sources will already have widely covered the rapidly rising stocks, so that interpersonal communications are unnecessary to bring attention to them.

Our figures in section~\ref{subsec:shillerpound} reflect the paper's median estimates (of infection and removal rates) for individual and for institutional investors, and for randomly selected versus for rising stocks, respectively\footnote{We convert all the continuous-time rates into discrete-time and from annual to weekly frequency. For instance, the recovery rate estimated from the decaying pattern of the time spent on studying a given stock for INSRPI is $g=1.39$ (a half-life of $ln(2)/g=0.50$ years). In discrete-time and at weekly frequency, this is equivalent to a probability of recovery $\gamma = 1-\exp^{-g/52} =0.02$. For the removal rate, under the assumption made by the paper that the fraction of susceptible is close to $1$ despite being time-varying, the estimated median removal rate of INSRPI is $b = 2.02$. It is converted to a weekly probability of $\beta = 1-\exp^{-b/52}=0.038$.}.  In addition, we set the initial fraction of the infected to be $1$ percent.

[Insert Figure \ref{fig:sir_simulate} here]

A first point to make in our economic analysis is that the epidemiological analysis above is for parameters that characterize a set of people who are highly interested and motivated investors.  There is no sense in which these parameters can be thought of as characterizing the whole population -- which is why it is not as surprising or implausible as it might at first have appeared that all the parameterizations of the models were ones in which $R$ (the proportion of investors who would eventually become interested in a stock) was high.

The economic analysis can now also be interpreted in temporal terms.  It takes around half a year for the interest of institutional investors to reach its peak and a little more than a year for a rapidly rising stock. As for individual investors, the interested population reaches its peak after 40 weeks for a random stock and 2.5 years for a rapidly rising stock.

%sAlthough such paths correspond to a particular set of configurations, they presumably shed light on the general evolution patterns of stock investors' interests.  For instance, so long as the infection rate is higher than the removal rate, the fraction of infected investors will follow a hump shape  -- because the number of susceptible persons is declining.

One insight the authors draw from epidemiological models is that if the interest spreads among investors from one to another gradually, investors' expectations and decision responses should not all bunch around dates of news events.

The paper also argues that if the infection rate is close to the removal rate, an implication might be that stock prices follow a random walk.  This would be an example of an economic consequence flowing from the pattern of spread of the infection.

Remarkably little of the extensive literature citing \cite{shiller1989survey} has involved meaningful epidemiological modeling; almost all of it has either been empirical, or has used a modeling framework that cannot be characterized as `epidemiological' as we are interpreting the term (see above).

A potential reason for this lack of followup is the nonexistence, until quite recently, of much direct data on either of the two key components of the model: beliefs (about, say, stock prices), or social connections -- and no data at all about the \textit{changes} of beliefs as a function of the structure of a measured social network.  \cite{shiller1989survey} had to make heroic assumptions in order to quantify their model.  Few subsequent scholars seem to have been willing to go so far in employing what might today be termed an `indirect inference' approach: ``Assuming the epidemiological model is right, let's calibrate it using its downstream implications for things we can observe.''

However, there are two good exceptions, both of which estimate parameters of structural epidemiological model of stock investors using microdata.

The first is  \href{https://github.com/iworld1991/EpiExp/blob/master/Literature/shive2010epidemic.pdf}{\cite{shive2010epidemic}}, which uses an SI (`susceptible-infected') model to inform the structure of a reduced-form regressor that aims to capture social influences among investors.  Using nearly the universe of ownership data for Finnish stocks between 1994 and 2004, the author assumes that the key social infection channels are at the municipal level, and estimates the time-series dynamics of ownership within municipalities.

Specifically, controlling for all of the variables (demographic variables, news sources, price dynamics, and others) that standard models in economics and finance might suggest could affect ownership patterns, the author estimates an equation that can be interpreted as measuring the $\beta$ coefficient in our \eqref{eq:sirdyn} model above.  The estimated $\beta$ coefficient is highly statistically significant, indicating at a minimum that there is some local dynamic pattern to stock purchases not captured by the usual finance and economic models, but which is captured by `proportion locally infected last period.'  An epidemiological interpretation of this fact seems quite natural.

The second example is \href{https://github.com/iworld1991/EpiExp/blob/master/Literature/huang2021rate.pdf}{ \cite{huang2021rate}}, which estimates an epidemiological model of diffusion of financial news among geographically neighbors. The paper reports a time-average estimate of the reproduction ratio $\mathcal{R}$ between $0.3$ to $0.4$ (equivalent to $\frac{\beta S_t/N}{\gamma}$ in a SIR model); that is, each stock trade that the authors identify as exogenous (see the paper for the mechanism) resulted in a total of $0.3{\sim}0.4$ trades among that person's neighbors, aggregated over all neighbors and all time.

The authors also find stronger transmission between investors of the same characteristics (age, income category, and gender, confirming the usual presumption of homophily -- people tend to trust others with similar backgrounds). Also interesting for the epidemiological modeling, the paper found stronger transmission between senders and receivers with high past performance, suggesting that conversations between neighbors were more likely when past performance has been high.  The natural interpretation -- consistent with common findings in behavioral finance -- is that you are more likely to mention your investment in a winner than a loser.

Their estimate that $\mathcal{R}$ is positive and highly statistically significant is consistent with the presence of neighborly social influence, and they work hard to rule out plausible alternatives. But since the estimated reproduction ratio is below the 1, their results imply that news of this kind does not lead to an epidemic of stock trading.  This is in contrast with \cite{shiller1989survey} work, whose corresponding reproduction ratios far exceeded one.  This difference highlights the extent to which epidemiological models must be interpreted with care; even if similar phenomena (stock trading) are being studied, and even if there is evidence of social communication, the estimated nature and size of the epidemiological consequences can vary greatly depending on the exact experiment.

A final, and very impressive, contribution that satisfies all our criteria is a model of housing market fluctuations by \href{https://www.journals.uchicago.edu/doi/abs/10.1086/686732}{\cite{burnside_understanding_2016}}, which shows how incorporating social interactions can generate booms and busts. In the model, agents not only differ in their belief (optimistic or skeptical) about the fundamental value of housing, they also differ in their degree of confidence in their own beliefs.  Theirs is a random mixing model in the sense that agents randomly meet each other, but the paper has a mechanism that is similar in its implications to the simplest epidemiological model of `super-spreaders' in which some agents have many more social connections than others:  The agents in this model differ in the degree of confidence they have in their opinions (whether optimistic or pessimistic) and those with greater confidence in their beliefs are more likely to convert those who have less confidence.  Defining a `boom' as a period in which house prices rise rapidly as the result of a spread in optimism, and a `bust' as a rapid decline in prices caused by rising proportion of skepticism, their most interesting result is that whether a boom is followed by a bust can depend on whose opinion (optimists or skeptics) turns out to be closer to the true fundamental value.  Specifically, busts happen when the skeptics turn out to be about the fundamentals, while booms that are caused by optimists who happen to be right are not followed by busts.

