
\subsection{Contagion}\label{subsec:Contagion}

In in the epidemiology literature and in ordinary usage the word ``contagion'' means essentially `epidemic of a transmissible disease.'  And large literatures in economics and finance describe themselves as investigating the phenomenon of `contagion' as it applies in those fields.  But for reasons we articulate here, most of this work is quite different from what we define as an EE modeling approach.

\subsubsection{Multiple Equilibrium}\label{subsubsec:multipleEqulibrium}

The canonical paper in the literature on `bank runs' is \href{https://www.jstor.org/stable/1837095}{\cite{diamond_bank_1983}}, who construct a model with two RE (self-fulfilling) equilibria.  In one, all depositors attempt to withdraw their savings from the bank and it fails; in the other nobody wants to withdraw their savings and the bank remains sound.  But the paper's model fails our first criterion for an EE model because there is no dynamic process by which the ideas `spread' and has no measurable implications for expectational dynamics at either the micro or the macro level.

%In economics as well as finance, much of the theoretical work that is cited in the literature on `contagion' is also really about multiple equilibria.  \cite{brock2005social} construct a model in which interdependence of agents expectations can lead to multiple equilibria

Much of the theoretical work in economics and finance that describes itself as being about `contagion' is of this kind -- that is, about multiple equilibria without any description of transmission or dynamics.

There is nothing intrinsic about such questions that prohibits the construction of what we would call a genuinely epidemiological model -- indeed, work by \cite{iyer2012understanding} makes an excellent start by collecting data on detailed dynamics of bank withdrawals among members of a social network during a bank run episode.  The authors write:
``we want to understand ... contagion in bank runs. In order to model this, we draw on a long, time honored literature on contagion of infectious diseases in the epidemiology literature.''  They proceed to note that ``the parallel [to infection] in bank runs is the probability of running as a result of contact with a person who has already run.''

But after estimating this infection rate using their social network data, they stop without specifying the other elements required to define or simulate a full epidemiological model.  (Though these would be interesting steps to pursue for someone interested in advancing this agenda).

These authors felt it necessary to be explicit that they were invoking the meaning of `contagion' from the epidemiology literature to distinguish their ideas from those explored in the large literature on `financial contagion.'

Defining exactly what is meant by financial contagion has been a challenge for this literature (\cite{pericoli2003primer}),  but none of the usual definitions correspond at all closely to the epidemiological perspective that the way to model a contagion is to understand the microscopic channels by which an idea is transmitted from actor to actor and to use those mechanisms to analyze the circumstances under which it will spread.  Instead, financial contagion was given an influential early definition as ``the spread of market disturbances — mostly on the downside —
from one country to the other,'' by~\cite{Dornbusch00contagionHow} after the Asian Financial Crisis of 1997 prompted a great many studies examining questions like the time series correlations of asset price movements in the affected countries.
%\begin{comment}
%http://www1.fee.uva.nl/fm/papers/Claessens/Contagion_WBRO.pdf
%\end{comment}.

But one branch of the literature that developed after the financial panic that followed the collapse of Lehmann Brothers in 2008 is that markets can be vulnerable to the sudden disappearance of entities that are `too interconnected to fail.' This interpretation led to a literature that examined datasets on the interconnections between financial institutions, using many of the same tools (network theory, random graphs, etc) that have been used to model the transmission of ideas across social networks.

In this work, what is modeled as being transmitted along the network connections is usually financial flows (rather than ideas or expectations), and the modeled mechanisms of contagion involve consequences of disruptions to those flows.  Despite the overarching ``contagion'' metaphor, the low-level elements of the transmission process generally do not have immediate interpretations corresponding to epidemiological primitives like `infectiousness,' and the literature does not mainly aim to model the dynamics of expectations at either the micro or the aggregated level.  (See~\cite{glasserman2016contagion} for a summary of this literature and~\cite{cabrales2015financial} for a deep dive).

It is possible that some of this work could be reinterpreted to fit into our definition of EE modeling, in the same way that the work on technology diffusion
clearly fits our definitions and has a straightforward interpretation (already anticipated by~\cite{arrow_classificatory_1969}).  But the literature is so vast and complex, and the reinterpretation would have to be so thorough, that this is a task we hope might be undertaken by others who want to bring the insights from that literature to a new audience who might be more receptive if the ideas were repackaged.

\begin{itemize}

	\item
	bank runs/spread of panic and fear

	\begin{itemize}
		\item
		canonical models are basically timeless: run happens instantly
		\href{https://www.jstor.org/stable/1837095}{\cite{diamond_bank_1983}}
		\item also,  the run arises as one of the multi-equilibra
		\item
		in reality, both the process and the outcome are likely driven by how information/fear spreads across the social network
		\begin{itemize}
			\item  the unfolding of a bank run using high-frequency data on deposits withdrawing and social network: \href{https://www.aeaweb.org/articles?id=10.1257/aer.102.4.1414}{\cite{iyer2012understanding}}
			\item runs are more likely to diffuse with similar bank/community characteristics,(suggesting infection rate is not constant in epi models);  \href{https://journals.sagepub.com/doi/abs/10.1177/0003122416629611}{\cite{greve2016ripples}};
			\item  depositors who learned from acquaintances about the bad news regarding banks first closed bank accounts; \href{https://www.aeaweb.org/articles?id=10.1257/aer.90.5.1110}{\cite{kelly2000market}}
			%\item other experimental evidences:\href{https://www.sciencedirect.com/science/article/pii/S0167268114000341}{\cite{kiss2014social}}, \href{https://ideas.repec.org/a/eee/beexfi/v20y2018icp115-130.html}{\cite{shakina2018coordination}}, \href{https://fr.wikipedia.org/wiki/Panique_bancaire}{\cite{dijk2017bank}}
		\end{itemize}
		\item
		financial crisis in the Great Recession has been described as
		``giant extended bank run on financial sector''
	\end{itemize}


\end{itemize}
